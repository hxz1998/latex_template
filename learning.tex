% -*- coding: utf-8 -*-
% learning.tex
% 勾股定理
\documentclass[UTF8]{ctexart}

\title{杂谈勾股定理}
\author{张三}
\date{\today}

\bibliographystyle{plain}

\begin{document}

\maketitle
\tableofcontents
\begin{abstract}
    
\end{abstract}
\section{勾股定理在古代}
勾股定理是一个基本的几何定理,指直角三角形的两条直角边的平方和等于斜边的平方。中国古代称直角三角形为勾股形,并且直角边中较小者为勾,另一长直角边为股,斜边为弦,所以称这个定理为勾股定理,也有人称商高定理。

勾股定理现约有 500 种证明方法,是数学定理中证明方法最多的定理之一。勾股定理是人类早期发现并证明的重要数学定理之一,用代数思想解决几何问题的最重要的工具之一,也是数形结合的纽带之一。在中国,周朝时期的商高提出了“勾三股四弦五”的勾股定理的\emph{特例}。在西方,最早提出并证明此定理的为公元前 6 世纪古希腊的\footnote{毕达哥拉斯学派},他用演绎法证明了直角三角形斜边平方等于两直角边平方之和。

答周公问:
\begin{quote}
\zihao{-5}\kaishu 勾广三,股修四,径隅五。
\end{quote}
\section{勾股定理的近代形式}
\bibliography{math}

\end{document}